\input{theme}

\usepackage{amsmath}
\usepackage{booktabs}
\usepackage{rotating}
\usepackage{wasysym}

\graphicspath{{images/}}

\title[l-QCD vs. spacetime]
    {Putting l-QCD on a dynamic spacetime}
\author[Justinas Benikasas]{
    \texorpdfstring{\underline{Justinas Benikasas}$^*$}{Justinas Benikasas},
    \texorpdfstring{Dr. Chris Bouchard}{}}
\institute{University of Glasgow\\}
\date{Mar 12, 2021 | Glasgow, Scotland\vspace{6pt}}

\newcommand{\yes}{\CIRCLE}
\newcommand{\no}{\Circle}
\newcommand{\partially}{\LEFTcircle}

\begin{document}

\begin{frame}[plain]
  \titlepage{}
\end{frame}

\begin{frame}[t]{Table of contents}
\only<1-3>{
    \begin{textblock*}{\paperwidth} (0.05\paperwidth,25pt)
        \begin{enumerate}
            \item<1-> \textbf{Introducing lattice quantum chromodynamics (l-QCD)}
                \begin{itemize}
                    \item 
                \end{itemize}
            \item<2-> \textbf{Implementing semi-classical GR}
                \begin{itemize}
                    \item Discrete spacetime continuum
                \end{itemize}
        \end{enumerate}
        \only<3>{
        \hspace*{6cm}
        \includegraphics[width=250pt]{Lattice - Einstein Hilbert.png}
        }
    \end{textblock*}
}
% \only<4-5>{
%     \begin{textblock*}{\paperwidth} (0.05\paperwidth,40pt)
%         \textbf{Our contribution:}
%         \begin{enumerate}
%             \itemsep5pt
%             \item Formal approach to\\
%             \textbf{collaborative real-time feature modeling}\\
%             with focus on
%             \begin{itemize}
%                 \item Consistency maintenance
%                 \item Conflict detection \color{gray}{(\& resolution)}
%             \end{itemize}
%             \only<5>{\item Open-source research prototype}
%         \end{enumerate}
%     \end{textblock*}
% }
% \begin{textblock*}{0.37\paperwidth} (0.58\paperwidth,30pt)	
%     \includegraphics<5>[width=\linewidth]{tool}
% \end{textblock*}
\end{frame}

% \begin{frame}[t]{l-QCD}
% \begin{textblock*}{0.54\paperwidth} (0.025\paperwidth,5.5pt)
%     \includegraphics<1>[width=\linewidth]{conflict}
%     \centering
%     \textbf{Formal description}
% \end{textblock*}
% \begin{textblock*}{0.38\paperwidth} (0.588\paperwidth,10pt)
%     \includegraphics<1>[width=\linewidth]{tool}
%     \centering
%     \textbf{Open-source prototype}
% \end{textblock*}
% \end{frame}

\begin{frame}[t]{Introducing L-QCD}
\begin{textblock*}{\paperwidth} (0.05\paperwidth,25pt)
\begin{itemize}
    \item[] <1-2>\textbf{What is a lattice and why do we need it?}
    \item[] <2>\textbf{Performing calculations on the lattice}
\end{itemize}
\end{textblock*}
\end{frame}

\begin{frame}[t]{Introducing L-QCD}
\begin{textblock*}{\paperwidth} (0.05\paperwidth,25pt)

\begin{itemize}
    \item[] <1-4>\textbf{What is a lattice and why do we need it?}
        \begin{itemize}
            \item <2-4> Working with minimal energies
            \item <3-4> Computationally easier to implement
        \end{itemize}
\end{itemize}

\vspace{-1.8cm}{
\hspace{8.5cm}
% \includegraphics<4>[width=175pt]{Lattice.png}
}
\end{textblock*}
\end{frame}

\begin{frame}{Introducing L-QCD}
    \item[] <1-4> \textbf{Performing calculations on the lattice}
\end{frame}

\begin{frame}[t]{Concurrency Control}
\begin{textblock*}{\paperwidth} (0.125\paperwidth,20pt)
    % \begin{tabular}{rllllllll}
    %     \toprule
    %     \vspace{1.5cm}\\
    %     & \begin{rotate}{50} \textbf{Turn-Taking} \end{rotate} \hspace{0.5cm}
    %     & \begin{rotate}{50} \textbf{Locking} \end{rotate} \hspace{0.5cm}
    %     & \begin{rotate}{50} \textbf{{CRDTs}} \end{rotate} \hspace{0.5cm}
    %     & \begin{rotate}{50} \textbf{Serialization} \end{rotate} \hspace{0.5cm}
    %     & \begin{rotate}{50} \textbf{{OT}} \end{rotate} \hspace{0.5cm}
    %     & \begin{rotate}{50} \textbf{{MVSD}} \end{rotate} \hspace{0.5cm}
    %     & \begin{rotate}{50} \textbf{{MVMD}} \end{rotate} \hspace{0.5cm}
    %     & \\
    %     \midrule
    %     \textbf{Concurrency} & \no & \partially & \yes & \yes & \yes & \yes & \yes \\
    %     \textbf{Optimism} & \partially & \no & \yes & \yes & \yes & \yes & \yes \\
    %     \textbf{Intention Preservation} & \yes & \no & \no & \no & \no & \partially & \yes \\
    %     \textbf{Flexibility} & \yes & \no & \no & \no & \no & \no & \partially \\
    %     \textbf{Correctness} & \yes & \partially & \partially & \yes & \no & \no & \partially \\
    %     \bottomrule
    % \end{tabular}
\end{textblock*}
\end{frame}

\end{document}